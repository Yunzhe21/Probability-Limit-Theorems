\documentclass[12pt]{article}
\usepackage[margin=1in]{geometry}
\usepackage[all]{xy}


\usepackage{amsmath,amsthm,amssymb,color,latexsym}
\usepackage{geometry}        
\geometry{letterpaper}    
\usepackage{graphicx}

\newcommand{\legendre}[2]{\ensuremath{\left( \frac{#1}{#2} \right) }}
\newtheorem{problem}{Problem}

\newenvironment{solution}[1][\it{Solution}]{\textbf{#1. } }{$\square$}


\begin{document}
\noindent Probability Limit Theorems \hfill Assignment 1\\
Yunzhe Zheng. (2025/09/28)

\hrulefill

\begin{problem}
    Show that the following sets of subsets of $\mathbb{R}$ generates the same $\sigma$-algebra:
    $$
        B:=\{(a, b): a<b\}, B_1:=\{(a,b]: a<b\}, B_2:=
        \{(-\infty, b]: b\in\mathbb{R}\}
    $$
\end{problem}

\textbf{Proof:} To show that $\mathcal{B}:=\sigma(B)=\sigma(B_1)=:\mathcal{B}_1$. For any $(a,b)\in B$, $(a, b)=\bigcup_{n}(a, b-\frac{1}{n}]$, where $n$ choose sufficiently large such that $a<b-\frac{1}{n}$, thus $\mathcal{B}\subseteq\mathcal{B}_1$. Conversely, For any $(a,b]\in B_1$, $(a,b]=\bigcap_{n\in\mathbb{N}}\left(a, b+\frac{1}{n}\right)$, then $\mathcal{B}_1\subseteq\mathcal{B}$, hence $\mathcal{B}=\mathcal{B}_1$.\\ 
\indent To show $\mathcal{B}_1=\mathcal{B}_2$. Similarly, for any $(a,b]\in B_1$, $(a,b]=(-\infty, b]\cup(-\infty, a]^c$, thus $\mathcal{B}_1\subseteq\mathcal{B}_2$. Conversely, for $(-\infty, b]\in B_2$, $(-\infty, b]=\bigcup_n (-n, b]$, where $-n$ is chosen such that $-n<b$, thus $\mathcal{B}_2\subseteq\mathcal{B}_1$, hence $\mathcal{B}_1=\mathcal{B}_2$. \qed
\\
\begin{problem}
    Show that every open subset $G$ of $\mathbb{R}$ is a disjoint countable union of open intervals.
\end{problem}

\textbf{Proof:} For fixed $x\in G$, consider the largest interval such that it is contained in $G$, namely $I_x=(a, b)$ where 
$$
    a =\inf\{y: y<x, y\in G\}, \ b=\sup\{z: x<z,z\in G\}
$$ 
then $G=\bigcup_{x\in G}I_x$. Notice that for $I_x,I_y$ defined as above for $x\neq y$ such that $I_x\cap I_y\neq\emptyset$, for $x$, $I_x\cup I_y$ would be a larger interval containing $x$ and contained in $G$, hence $I_x=I_y$, indicating that $\{I_x: x\in G\}$ are collection of disjoint open intervals automatically. Finally, we are left to show that it is countable union. Indeed, for each $I_x$ (assuming the Axiom of Choice implicitly makes the specification valid), it contains a rational number, then we create a one-to-one correspondence between $\{I_x:x\in G\}$ and a subset of $\mathbb{Q}$, suggesting that is indeed a countable union. \qed
\\ 
\begin{problem}
    Let $\tau$ be the set of all open subsets of $\mathbb{R}$, and let the $\pi$-system
    $$
        \pi(\mathbb{R}):=\{(-\infty,x]: x\in\mathbb{R}\}
    $$ Deduce from Problem 1 and 2 that $\mathcal{B}(\mathbb{R}):=\sigma(\tau)=\sigma(\pi(\mathbb{R}))$.
\end{problem}

\textbf{Proof:} Thanks to conclusion in Problem 1, $\sigma(\pi(\mathbb{R}))=\sigma(\{(a, b): a<b\})$, then we are left to show that $\sigma(\tau)=\sigma(\{(a,b): a < b\})$. $\sigma(\{(a,b): a<b\})\subseteq \sigma(\tau)$ is immediate since $\{(a,b): a<b\}\subseteq \tau$. Conversely, applying conclusion in Problem 2, we see that for any open subset $G\in\tau$, $G=\bigcup_{i=1}^{n}I_i$ for disjoint sequence of open intervals $\{I_i\}$, thus $\sigma(\tau)\subseteq \sigma(\{(a, b): a<b\})$. To conclude, $\mathcal{B}(\mathbb{R})=\sigma(\pi(\mathbb{R}))$. \qed
\\
\begin{problem}
    Let $\mu$ be a finite-valued, additive set function on an algebra $\mathcal{A}$. Show that $\mu$ is countably additive iff for any sequence $\{A_n\}_{n\in\mathbb{N}}\subseteq\mathcal{A}$ with
    $$
        A_n\supseteq A_{n+1} \ \forall n\in\mathbb{N}, \ \bigcap_{n\in\mathbb{N}}A_n=\emptyset\implies \mu(A_n)\to0\text{ as } n\to\infty.
    $$
\end{problem}

\textbf{Proof:} Suppose that $\mu$ is countably additive, and consider $\{A_n^c\}$, then $A_n^c\subseteq A_{n+1}^c$. Define $B_i=A_{i+1}^c\setminus A_i^c$, then $\{B_i\}$ are disjoint, and 
\begin{align*}
    \mu\left(\bigcup_i A_{i}^c\right)=\mu\left(\bigcup_i B_i\right)=\sum\limits_i\mu\left(B_i\right)&=\lim\limits_{n\to\infty}\left(\sum\limits_{i=2}^n\left(\mu(A_i^c)-\mu(A_{i-1}^c)\right)+\mu(A_1^c)\right) \\
    &= \lim\limits_{n\to\infty} \mu(A_n^c)
\end{align*}
Now, consider $E_i=A_1\setminus A_i$, then $E_n\subseteq E_{n+1}$
$$
    \mu\left(\bigcup_i E_i\right)=\lim\limits_{i\to\infty}\mu\left(E_i\right)=\lim\limits_{i\to\infty}(\mu(A_1)-\mu(A_i))=\mu(A_1)-\lim\limits_{i\to\infty}\mu(A_i )
$$
while at the same time, 
$$
    \mu\left(\bigcup_i E_i\right)=\mu\left(A_1\cap \left(\bigcap_i A_i\right)^c\right)=\mu(A_1)-\mu\left(\bigcap_{i=1}^\infty A_i\right)
$$ 
hence 
$$
    0=\mu\left(\bigcap_{i=1}^\infty A_i\right)=\lim\limits_{i\to\infty}\mu(A_i)
$$
\\
\indent Conversely, for $\{E_i\}$ collection of disjoint sets such that $\bigcup_{i=1}^\infty E_i\in\mathcal{A}$, consider the collection $\{A_n\}$ where $A_n=\left(\bigcup_{i=1}^\infty E_i\right)\setminus\left(\bigcup_{i=1}^nE_i\right)$, then obviously $A_n$ is decreasing and $\bigcap_{i=1}^\infty A_i=\emptyset$, then 
\begin{align*}            \lim\limits_{n\to\infty}\mu\left(A_n\right)&=\lim\limits_{n\to\infty}\mu\left(\bigcup_{i=1}^{\infty}E_i\setminus\bigcup_{i=1}^{n}E_i\right)=\mu\left(\bigcup_{i=1}^\infty E_i\right)-\lim\limits_{n\to\infty}\mu\left(\bigcup_{i=1}^nE_i\right) \\
&=\mu\left(\bigcup_{i=1}^\infty E_i\right)-\sum\limits_{i=1}^\infty\mu\left(E_i\right)=0
\end{align*}
then we obtain the countable additivity. \qed
\\
\begin{problem}
    Given $\alpha<\beta$, let $\mathcal{A}$ be the algebra of finite unions of disjoint intervals of the form 
    $$
        A=(a_1, b_1]\cup\cdots\cup(a_n,b_n], \ \alpha\leq a_k,b_k\leq \beta, \ 1\leq k\leq n
    $$
    and define, for every $A\in\mathcal{A}$, $\mu(A):=\sum\limits_{i=1}^n(b_i-a_i)$. \\
    \indent (i). Check that $\mu$ is well-defined. \\
    \indent (ii). Prove, using Problem 4, that $\mu$ is countably additive on $\mathcal{A}$.
\end{problem}

\textbf{Proof:} (i). If $A:=\bigcup\limits_{i=1}^n(a_i, b_i]=\bigcup_{i=1}^m(a'_i, b'_i]=: A'$, then we can obtain a refined partition $\alpha\leq p_1<p_2<\cdots<p_N\leq \beta$, where $p_i$ are endpoints of $A$ and $A'$ such that $A=A'=\bigcup\limits_{i=1}^N(p_i,p_{i+1}]$
$$
    \mu(A)=\sum\limits_{i=1}^N(p_{i+1}-p_{i})=\mu(A')
$$
which proves that $\mu$ is well-defined. \\
\indent (ii). First, finite additivity is immediate from definition of $\mu$, suppose we have collection of disjoint $\{E_i\}_{i=1}^n$ in $\mathcal{A}$, each of the form $E_i=\bigcup\limits_{i=1}^{n_i}(a_{i, k}, b_{i, k}]$, then 
$$
    \mu\left(\bigcup_{i=1}^n E_i\right)=\sum\limits_{i=1}^n\sum\limits_{k=1}^{n_i}(b_{i,k}-a_{i,k})=\sum\limits_{i=1}^n\mu(E_i)
$$
\\
\indent Now suppose we have a decreasing sequence $\{A_n\}_{n\in\mathbb{N}}\subseteq\mathcal{A}$ such that $\mu(A_n)\geq\epsilon$ for some $\epsilon>0$, we aim to find a decreasing bounded subsets $B_n\in\mathcal{A}$ such that $\overline{B_n}\subseteq A_n$ with $\bigcap_{n\in\mathbb{N}}B_n\neq\emptyset$. For each $A_n=\bigcup\limits_{i=1}^{m_n}(a_{n,i}, b_{n, i}]$, we can shrink each interval by a small margin to $C_n=\bigcup\limits_{i=1}^{m_n}(a_{n,i}+\delta_{n,i},b_{n,i}]$ such that $\mu(A_n\setminus C_n)\leq \epsilon/2^{n+1}$, and define $B_n=\bigcap\limits_{i=1}^n C_i$, then $B_n$ is decreasing sequence of bounded subsets, we also have 
$$
    \mu(A_n\setminus B_n)\leq \mu\left(\bigcup_{i=1}^n(A_n\setminus C_i)\right)\leq \mu\left(\bigcup_{i=1}^n(A_i\setminus C_i)\right)\leq \sum\limits_{i=1}^\infty \epsilon/2^{i+1}=\epsilon/2
$$
then 
$$
    \mu(B_n)=\mu(A_n)-\epsilon/2\geq \epsilon/2
$$
Also, for each $B_n$, $\overline{B_n}\subseteq A_n$, since each $\overline{C_n}\subseteq A_n$. $\{\overline{B_n}\}$ is the collection of decreasing compact sets, then the intersection of them is nonempty, thus $\bigcap\limits_{i=1}^\infty A_n$ is nonempty. Finally, we can apply Problem 4 to obtain that $\mu$ is countable additive. \qed  
\\
\begin{problem}
    Using Problem 5, prove that there exists, for all $\alpha, \beta\in\mathbb{R}$, a unique Borel measure $\mu$ on $(\alpha,\beta]$ such that, for all $a,b\in (\alpha, \beta]$ with $a<b$, 
    $$
        \mu((a,b])=b-a
    $$
    Prove also there exists a unique such measure on the whole of $\mathbb{R}$
\end{problem}

\textbf{Proof:} From Problem 5 we've prove that $\mu$ is countable additive on algebra $\mathcal{A}$ with $\mu(\mathcal{A})=\beta-\alpha<\infty$, then by Caratheodory's extension theorem, there exists a unique Borel measure on $(\alpha,\beta]$. \\
\indent For the case of $\mathbb{R}$, we consider the algebra $\mathcal{A}$ generated by 
$$
    \varepsilon=\begin{cases}
  (a,b] & \text{ for } -\infty\leq a<b<+\infty \\
  (a,+\infty) & 
\end{cases}
$$
and the measure on $\mathcal{A}$ is defined similarly, $\mu((a,b])=b-a$, $\mu((a, +\infty))=\infty$, and $\mu((-\infty, a])=\infty$, then we are left to prove that $\mu$ is countable additive. Suppose we have $\{A_i\}$ collection of disjoint elements in $\mathcal{A}$ and that $A:=\bigcup_{i=1}^{\infty}A_i\in\mathcal{A}$, then if $A$ is unbounded, then the countable additivity is immediate from definition, otherwise, $A=\bigcup_{i}^n(a_i,b_i]$, then by the countable additivity for $\mu$ restricted on bounded $A$ with finite measure, we conclude that $\mu$ is indeed countable additive. Then by Caratheodory extension theorem, there exists a measure. \\
\indent Further, since $\varepsilon\cup\{\emptyset\}$ is a $\pi$-system, the measure is indeed unique. \qed
\\   
\begin{problem}
    Given $A_1, \dots, A_m$ subsets of a nonempty set $E$, let $\mathcal{A}:=\{A_1, \dots, A_m\}$ and let $\varepsilon:=\sigma(\mathcal{A})$. For $\epsilon\in\{0,1\}^m$, define $A^\epsilon:= A_1^{\epsilon_1}\cap\cdots\cap A_m^{\epsilon_m}$ where $A_k^0=A_k^c$, $A_k^1=A_k$. \\
    \indent (i). Prove that 
    $$
        A\in\varepsilon\iff A=\bigcup_{\epsilon\in I}A^\epsilon \text{ for some } I\subseteq\{0,1\}^m
    $$ \\
    \indent(ii). Deduce that, for all $\epsilon\in\{0,1\}^m$ and $A\in\varepsilon$, either $A\cap A^\epsilon=\emptyset$ or $A^\epsilon\subseteq A$. The subsets $A^\epsilon$ are called atoms of $\varepsilon$. \\
    \indent (iii). Deduce that, if $f:(E,\varepsilon)\to(\mathbb{R}, \mathcal{B}(\mathbb{R}))$ is measurable, then there exist $a_\epsilon\in\mathbb{R}$, $\epsilon\in\{0,1\}^m$, such that 
    $$
        f=\sum\limits_{\epsilon\in\{0,1\}^m}a_\epsilon\textbf{1}_{A^\epsilon}
    $$
\end{problem}

\textbf{Proof:} (i). The direction from right to left is trivial from the definition of $\sigma$-algebra. Conversely, we only need to show that the collection $\mathcal{C}=\{\bigcup_{\epsilon\in I}A^\epsilon: I\subseteq\{0,1\}^m\}$ is a $\sigma$-algebra. Indeed, $\emptyset=\bigcup_{\epsilon\in\emptyset}A^\epsilon$ and $E=\bigcup_{\epsilon\in\{0,1\}^m} A^\epsilon$. For $F=\bigcup_{\epsilon\in I}A^{\epsilon}$, $F^c=\bigcup_{\epsilon\in\{0,1\}^m\setminus I}A^\epsilon\in\mathcal{C}$, since for each $\epsilon_1\neq\epsilon_2$, $A^{\epsilon_2}\cap A^{\epsilon_2}=\emptyset$. Now for countable union, notice that since the cardinality of $\{0,1\}^m$ is finite, we are essentially dealing with finite union, and finite union is obviously satisfied by the definition. Finally, since $\varepsilon$ is the smallest $\sigma$-algebra generated by $\mathcal{A}$, left to right implication is proved. \\
\indent (ii). For $\epsilon\in\{0,1\}^m$ and $A\in\varepsilon$, we know from previous result that we may write
$$
    A=\bigcup_{\epsilon\in I} A^\epsilon \text{ for some } I\subseteq\{0,1\}^m
$$
If $\epsilon'\in I$, then $A^{\epsilon'}\subseteq\bigcup\limits_{\epsilon\in I}A^\epsilon=A$. Otherwise if $\epsilon'\notin I$, we know that, as we also have mentioned in previous proof, from definition if $\epsilon_1\neq\epsilon_2\in \{0, 1\}^m$ then $A^{\epsilon_1}\cap A^{\epsilon_2}=\emptyset$, that's because there exists some $A_i$ such that $A^{\epsilon_1}\subseteq A_i$ and $A^{\epsilon_2}\subseteq A_i^c$. Thus $A\cap A^{\epsilon'}=\emptyset$. \\
\indent (iii). Since $f$ is measurable, then consider $H_r:=\{x\in E: f(x)>r\}\in\varepsilon$, then according to conclusion part (ii), we must have either $H_{r}\cap A^\epsilon=\emptyset$ or $H_{r}\supseteq A^\epsilon$. In other word, for any $x,y\in A^\epsilon$, $f(x),f(y)$ must be either both greater than $r$ or both less or equal to $r$, and since the choice of $r$ can be arbitrary, we have $f(x)=f(y)$, indicating that $f$ is constant on each atom of $\varepsilon$. Finally, since $A^\epsilon$ is a partition of $E$, we write 
$$
    f=\sum\limits_{\epsilon\in\{0,1\}^m}a_\epsilon\textbf{1}_{A^\epsilon}
$$
where $a_\epsilon$ is the constant value on atom $A^\epsilon$. \qed
\\
\begin{problem}
    (i). Suppose that $A_1, A_2, \dots$ are independent and $\mathbb{P}(A_n)<1$ for all $n\geq 1$. Prove that 
    $$
        \mathbb{P}(A_n\text{ i.o.})=1 \iff \mathbb{P}\left(\bigcup_{n=1}^\infty A_n\right)=1
    $$ \\
    (ii). Let $X_1, X_2, \dots$ be independent random variables. Prove that 
    $$
        \mathbb{P}\left(\sup_{n\geq 1}X_n<\infty\right)=1 \iff \sum\limits_{n=1}^\infty\mathbb{P}(X_n>a)<\infty
    $$
    for some positive real number $a$.
\end{problem}

\textbf{Proof:} (i). If $\mathbb{P}(A_n\text{ i.o.})=1$, then $\mathbb{P}\left(\bigcup\limits_{m=1}^\infty\bigcap\limits_{n\geq m}A_n^c\right)=0=\sum\limits_{m=1}^\infty\mathbb{P}\left(\bigcap\limits_{n\geq m}A_n^c\right)$, thus in particular we have $\mathbb{P}\left(\bigcap\limits_{n=1}^\infty A_{n}^c\right)=0$, hence $\mathbb{P}\left(\bigcup\limits_{n=1}^\infty A_n\right)=1$. Conversely, we have 
$$
    \mathbb{P}\left(\bigcup_{n=1}^\infty A_n\right)=1-\mathbb{P}\left(\bigcap_{n=1}^\infty A_n^c\right)=1-\prod\limits_{n=1}^\infty (1-\mathbb{P}(A_n))=1
$$
we can do approximation as follows, 
$$
    \prod\limits_{n=1}^\infty (1-\mathbb{P}(A_n))\leq \prod\limits_{n=1}^\infty e^{-\mathbb{P}(A_n)}=e^{-\sum\limits_{n=1}^\infty\mathbb{P}(A_n)}=0
$$
then $\sum\limits_{n=1}^\infty\mathbb{P}(A_n)=\infty$. By Second Borel-Cantelli lemma, $\mathbb{P}(A_n\text{ i.o})=1$. \\
\indent (ii). Suppose that $\sum\limits_{n=1}^\infty\mathbb{P}(X_n>a)=\infty$ for all $a$, then by Borel-Cantelli lemma ($\{X_n>a\}_n$ are independent since $\{X_n\}$ are independent), $\mathbb{P}(X_n>a\text{ i.o.})=1$. By the conclusion in Recitation 1, $\mathbb{P}\left(\bigcap\limits_{m=1}^\infty \{X_n>m\text{ i.o.}\}\right)=1$, then we have $\mathbb{P}(\sup_{n\geq 1} X_n=\infty)=1$, which leads to a contradiction. \\
\indent Conversely, Since $\sum\limits_{n=1}^\infty\mathbb{P}(X_n>a)<\infty$ for some real $a$, by First Borel-Cantelli lemma, we have $\mathbb{P}(X_n>a\text{ i.o.})=0$, then almost surely there exists a $N$ such that for $n\geq N$, $X_n\leq a$, then its immediate that  $\mathbb{P}\left(\sup_{n\geq 1}X_n<\infty\right)=1$.\qed 

\end{document}
