\documentclass[12pt]{article}
\usepackage[margin=1in]{geometry}
\usepackage[all]{xy}


\usepackage{amsmath,amsthm,amssymb,color,latexsym}
\usepackage{geometry}        
\geometry{letterpaper}    
\usepackage{graphicx}

\newcommand{\legendre}[2]{\ensuremath{\left( \frac{#1}{#2} \right) }}
\newtheorem{problem}{Problem}

\newenvironment{solution}[1][\it{Solution}]{\textbf{#1. } }{$\square$}


\begin{document}
\noindent Probability Limit Theorems \hfill Recitation 2\\
Yunzhe Zheng. (2025/09)

\hrulefill

\begin{problem}
    Let $\Omega=\mathbb{R}$ and let $\mathcal{F}$ be the collection of all $A\subseteq\Omega$ such that either $A$ or $A^c$ is countable. Define 
    $$
        \mathbb{P}(A)=\begin{cases}
  0& \text{ if } A \text{ is countable}\\
  1& \text{ if } A^c \text{ is countable}
\end{cases}
    $$
    Show that $(\Omega, \mathcal{F}, \mathbb{P})$ forms a probability space.
\end{problem}

\textbf{Proof:} (Step 1: Show that $\mathcal{F}$ is indeed a $\sigma$-algebra). Firstly, $\emptyset\in\mathcal{F}$ is trivial, and $\Omega\in\mathcal{F}$ since its complement $\emptyset$ is countable. Secondly, for any $F\in\mathcal{F}$, if $F$ is countable, then $F^c\in\mathcal{F}$ since $(F^{c})^c=F$ is countable; otherwise if $F^c$ is countable, then $F^c\in\mathcal{A}$ trivially. Finally, for sequence of disjoint $\{A_i\}\in\mathcal{F}^\mathbb{N}$, if all $A_i$'s are countable, there is nothing to prove, otherwise if for some $j\in\mathbb{N}$, $A_j^c$ is countable, then consider $\left(\bigcup\limits_{i=1}^\infty A_i\right)^c=\bigcap\limits_{i=1}^\infty A_i^c\subseteq A_j^c$ which is countable, hence the countable union is always in $\mathcal{F}$. \\
\indent (Step 2: Show that $\mathbb{P}$ is indeed a probability measure). First we have $\mathbb{P}(\emptyset)=0$ as $\emptyset$ is countable. Secondly, for a sequence of disjoint $\{A_i\}\in\mathcal{F}^\mathbb{N}$, we first observe that there can only be one of them, say $A_j$, such that $A^c$ is countable, for otherwise we have $A_j$ and $A_k$, $j\neq k$, such that $A_j^c,A_k^c$ are countable and $A_j\cap A_k=\emptyset$, then $(A_j\cap A_k)^c=A_j^c\cup A_k^c=\Omega=\mathbb{R}$, which is absurd since LHS is countable and RHS is uncountable. Therefore, we have for the latter case, $\left(\bigcup\limits_{i=1}^\infty A_i\right)^c=\bigcap\limits_{i=1}^\infty A_{i}^c\subseteq A_j^c$, which is countable, and it follows that
$$
    \mathbb{P}\left(\bigcup_{i=1}^\infty A_i\right)=1=\sum\limits_{i=1}^\infty\mathbb{P}(A_i)
$$
The former case is obvious from the fact that countable union of countable sets is still countable. Finally, immediately from the definition we have $\mathbb{P}(\Omega)=1$, concluding the proof. \qed
\\
\begin{problem}
    Let $(\Omega, \mathcal{F}, \mathbb{P})$ be a probability space and let $X$ be a measurable function. Define a set function $\mu$ on $(\mathbb{R}, \mathcal{B})$ by 
    $$
        \mu(B)=\mathbb{P}(X^{-1}(B)), \ B\in\mathcal{B}
    $$
    Show that $\mu$ is a probability measure. (Note: this measure is also called "image measure" or sometimes "pull-back measure".)
\end{problem}

\textbf{Proof:} Firstly, $\mu(\emptyset)=\mathbb{P}(X^{-1}(\emptyset))=\mathbb{P}(\emptyset)=0$ and $\mu(\mathbb{R})=\mathbb{P}(X^{-1}(\mathbb{R}))=\mathbb{P}(\Omega)=1$. Finally, for a sequence of disjoint $\{A_i\}_{i=1}^\infty\in\mathcal{B}^\mathbb{N}$, we know that $X^{-1}\left(\bigcup\limits_{i=1}^\infty A_i\right)=\bigcup\limits_{i=1}^\infty X^{-1}(A_i)$, then
$$
    \mu\left(\bigcup_{i=1}^\infty A_i\right)=\mathbb{P}\left(X^{-1}\left(\bigcup_{i=1}^\infty A_i\right)\right)=\mathbb{P}\left(\bigcup_{i=1}^\infty X^{-1}(A_i)\right)=\sum\limits_{i=1}^\infty\mathbb{P}(X^{-1}(A_i))=\sum\limits_{i=1}^\infty \mu(A_i)
$$
which shows that $\mu$ is indeed a probability measure. \qed
\\
\begin{problem}
    Prove that, if $(A_n)_{n\geq 1}$ are independent events, then
    $$
        \mathbb{P}\left(\bigcap_{n=1}^\infty A_n\right)=\prod\limits_{n=1}^\infty\mathbb{P}(A_n), \text{ and } \mathbb{P}\left(\bigcup_{n=1}^\infty A_n\right)=1-\prod\limits_{n=1}^\infty(1-\mathbb{P}(A_n))
    $$
\end{problem}

\textbf{Proof:} By the definition of independence, we have 
$$
    \mathbb{P}\left(\bigcap_{n=1}^N A_n\right)=\prod\limits_{n=1}^N\mathbb{P}(A_n)
$$
Notice that $\left\{\bigcap\limits_{n=1}^N A_n\right\}_N$ is a decreasing sequence converging to $\bigcap\limits_{n=1}^\infty A_n$, then by continuity of measure (Monotone Convergence), we have 
$$
    \mathbb{P}\left(\bigcap_{n=1}^\infty A_n\right)=\lim\limits_{N\to\infty}\mathbb{P}\left(\bigcap_{n=1}^N A_n\right)=\lim\limits_{N\to\infty}\prod\limits_{n=1}^N\mathbb{P}(A_n)=\prod\limits_{n=1}^\infty\mathbb{P}(A_n)
$$
Second equality holds immediately after what we just proved. \qed
\\
\begin{problem}
    Prove that 
    $$
        \limsup\limits_{n\to\infty}\liminf\limits_{k\to\infty}(A_n\cap A_k^c)=0
    $$
\end{problem}

\textbf{Proof:} From easy set theory we should see that $\limsup\limits_{n\to\infty}\liminf\limits_{k\to\infty}(A_n\cap A_k^c)=\limsup\limits_{n\to\infty}A_n\cap\liminf\limits_{k\to\infty}A_k^c=\emptyset$, as $(\limsup\limits_{n\to\infty} A_n)^c=\liminf\limits_{n\to\infty}A_n^c$. \qed

\end{document}